% !TEX program = xelatex -> bibtex -> xelatex*2

\documentclass[12pt]{ctexart}  
% ctexart支持中文编译文档
%最后全部翻译成英文后可以选择改成article,然后用pdflatex编译,还可以检验一下是不是没有中文了嘿嘿( •̀ ω •́ )✧
%官方要求字号不小于 12 号,此处选择 12 号字体

\CTEXsetup[format={\Large\bfseries}]{section} %在ctexart中,一级标题是居中的,这里改成左对齐

%% 本模板不需要填写年份,以当前电脑时间自动生成

%% 请在以下的方括号中填写队伍控制号
\usepackage[2615409]{ldmcm}  % 载入ldmcm模板文件
\problem{B}  % 请在此处填写题号

%%字体选择
%\usepackage{mathptmx}  % 这是 Times 字体,中规中矩 
\usepackage{mathpazo}  % 这是 COMAP 官方杂志采用的更好看的 Palatino 字体,可替代以上的 mathptmx 宏包

%%几处小修改,如无特殊需求可不做更改
\newcommand{\upcite}[1]{\textsuperscript{\textsuperscript{\cite{#1}}}}%这是参考文献引用上标的命令
\graphicspath{{img/}}          % 此处{img/}为相对路径,注意加上“/”
\let\itemize\compactitem
\let\enditemize\endcompactitem%解决列表环境中行距过大的问题

\title{Here is Your Title}  % 标题

% 如需要修改题头(默认为 MCM /ICM),请使用以下命令(此处修改为 MCM)
%\renewcommand{\contest}{MCM}

\usepackage{lastpage}  % 支持 \pageref{LastPage}

\usepackage{siunitx}

% ----------------------------------------------文档开始---------------------------------------------------------
\begin{document}

% 此处填写摘要内容-----------摘要摘要摘要摘要摘要摘要摘要摘要摘要摘要摘要摘要摘要摘要摘要摘要摘要摘要摘要摘要摘要
\begin{abstract}
	%第一段:2句话背景+2句话概括全文完成的任务

	Introduction of the background and the mission we accomplished.

	Global warming, El Niño... With the emergence of various extreme climates,\textbf{Austral-ia's wildfires} occur more frequently. The greenhouse gases emitted after combustion have exacerbated global warming, which seems to have entered an endless loop. At the same time, hundreds of millions of lives have been killed in the fire, which makes us sad. To better control wildfires, we modeled the \textbf{distribution of drones} assisting in the observation to achieve the best balance between economy and efficiency.

	%第二段:总结我们用了什么模型

	This is our models.

	Several models are established: Model I: Rasterized Multi-Objective Optimization Model; Model II: Model Verification Simulated by Poisson Process; Model III: Hovering Model Based on Tabu Search, etc.

	%接下来分别介绍模型与结果
	For Model I:
	Firstly,%首先,我们从哪里收集到了什么样的数据
	We find data。。。
	Then,%然后,基于什么样的原因,我们建立了什么模型
	we establish \textbf{model}。。。
	Next,%我们选择或者设计了什么算法
	we use Algorithm。。。
	Finally,%我们得到了什么样的结果
	%数值直接写出来,大量数据(表)或图片则说请参见(are shown in ...)图几
	%注意要用文字,不要使用符号
	it can be seen that。。。

	For Model II:
	Firstly,%首先,我们从哪里收集到了什么样的数据
	We find data。。。
	Then,%然后,基于什么样的原因,我们建立了什么模型
	we establish model。。。
	Next,%我们选择或者设计了什么算法
	we use Algorithm。。。
	Finally,%我们得到了什么样的结果
	%数值直接写出来,大量数据(表)或图片则说请参见(are shown in ...)图几
	%注意要用文字,不要使用符号
	it can be seen that。。。

	For Model III:
	Firstly,%首先,我们从哪里收集到了什么样的数据
	We find data。。。
	Then,%然后,基于什么样的原因,我们建立了什么模型
	we establish model。。。
	Next,%我们选择或者设计了什么算法
	we use Algorithm。。。
	Finally,%我们得到了什么样的结果
	%数值直接写出来,大量数据(表)或图片则说请参见(are shown in ...)图几
	%注意要用文字,不要使用符号
	it can be seen that。。。

	%灵敏度与稳健性分析
	Finally, sensitivity analysis 。。。 Meanwhile, robustness

	% 美赛论文中无需注明关键词。若一定要使用,
	% 请将以下两行的注释号 '%' 去除,以使其生效
	\vspace{5pt}
	\textbf{Keywords}: MATLAB, mathematics, LaTeX.

\end{abstract}

\maketitle  % 生成 Summary Sheet------------------------------------

\tableofcontents  % 生成目录


% -----------------------------------------正文开始-----------------------------------------------------------------------------------------------------------------------------------


%==============第一部分===引入=============================================================
\section{Introduction}
\subsection{Problem Background}%问题背景问题背景问题背景问题背景问题背景问题背景---------------------------------
%美赛一定要多上图,清晰直观,并且格式上最好是矢量图,例如pdf,而不是位图,例如jpg,png等,在形式上最好是组图,下面列出了利用\verb|subfigure|实现的
%\verb|1x2,1x3,2x2|的几种组图:
Some researchers envision an electrically powered Space Elevator System, which would provide a scalable 
infrastructure for interplanetary logistics, commerce, and exploration.

In its final operational configuration, the Space Elevator System will consist of three Galactic Harbours, 
preferably spaced 120 degrees apart along the equator. Each Galactic Harbour will include a single Earth port, 
connected to two 100,000 km-long tethers that link to two apex anchors. Multiple space elevators will operate 
collaboratively, each capable of transporting massive payloads from Earth to geosynchronous orbit (GEO) on a 
daily basis, and further to the apex anchors—where payloads can be loaded onto rockets for delivery to any destination with significantly reduced fuel consumption.
%% 这是问题背景的概念图
\begin{figure}[htbp]
	\centering
	\begin{subfigure}[b]{.8\textwidth} %设置缩放比例,这里的.5代表缩放为原来的50%
		\includegraphics[width=\textwidth]{img/background.png}
		\caption*{}
		\label{}
	\end{subfigure}
	\caption{problem background}\label{subfigure}
\end{figure}

Based on this proposed system, the Moon Colony Management (MCM) Agency intends to construct
 a lunar colony capable of accommodating 100,000 people, with construction scheduled to 
 commence in 2050 following the completion of the Space Elevator System. 
  Additionally, the agency is considering the use of traditional rockets to supply construction materials 
 and daily provisions to the lunar colony. Currently, there are ten rocket launch sites 
 on Earth, located in Alaska, California, Texas, Florida, and Virginia (United States), 
 as well as in Kazakhstan, French Guiana, the Satish Dhawan Space Centre (India), the 
 Taiyuan Satellite Launch Center (China), and the Mahia Peninsula (New Zealand).

As shown in figure\ref{subfigure},the schematic of the Space Elevator System and Moon Colony is illustrated above.

%\subsection{Problem Background}%问题重述与文献综述选一个------------------------------------------------------------------------
\subsection{Literature Review} % 文献综述-----------------
Three major problems are discussed in this paper, which are:
\begin{itemize}
	\item First and foremost, it is necessary to analyze three distinct scenarios for delivering 
	the 100 million metric tons of materials required for the construction of the 100,000-person 
	Moon Colony. The three scenarios to be considered include: 
	(a) relying solely on the three Galactic Harbors of the Space Elevator System for material delivery; 
	(b) relying exclusively on traditional rocket launches from selected existing space facilities (with 
	the flexibility to choose specific launch bases); 
	and (c) adopting a hybrid delivery strategy that combines the Space Elevator System’s Galactic Harbors and traditional rocket launches. 
	For each scenario, a comprehensive assessment of the material delivery feasibility, efficiency, and rationality must be conducted.

	\item Second, the robustness of the proposed material delivery solutions needs to be evaluated under 
	the condition that transportation systems are not in perfect working order. Potential anomalies may 
	include, but are not limited to, tether swaying of the Space Elevator System, rocket launch failures, 
	and elevator malfunctions. It is required to analyze the extent to which such operational imperfections 
	would alter the previously proposed delivery strategies and how the solutions should be adjusted to maintain the feasibility of material delivery.

	\item Third, the water demand of the 100,000-person Moon Colony over a one-year period after it is fully 
	operational must be systematically investigated. Based on the established material delivery model (from 
	the first research task), the additional costs and extended timeline required to ensure the colony has 
	sufficient water supply for one full year after its inhabitation should be quantified and analyzed, 
	thereby supplementing the delivery model with water supply-related constraints and considerations.

	\item Finally, the environmental impacts on Earth associated with achieving the 100,000-person Moon Colony 
	under each of the three material delivery scenarios must be discussed in depth. On this basis, corresponding 
	adjustments to the established delivery model should be proposed to minimize the adverse environmental impacts 
	on Earth while ensuring the successful construction and operation of the Moon Colony.
\end{itemize}

%月球水
A literatrue\upcite{jones2015water} reviews the water demand of lunar bases: the basic domestic water consumption 
is 4.17 kg per person per day, covering drinking, sanitation and other uses. A daily wastewater yield of 5.57 kg per person is generated, 
with condensed water and urinary wastewater recoverable as supplementary water sources. The base is designed with 167 kg of emergency 
water storage for a 10-day evacuation window, with low reliability requirements for water supply systems. Cost and reliability can be balanced via redundancy 
design or three-loop systems, and the relevant water system technologies are adaptable to the gravitational environments of the Moon and Mars (data based on NASA Exploration Program assumptions).

%太空电梯
A literature\upcite{smitherman2000space} confirm that space elevators have a solid technological and economic feasibility 
basis. High-strength materials such as carbon nanotubes meet the theoretical strength requirements; Low Earth Orbit (LEO) space elevators 
are technologically feasible with supporting propulsion solutions. Mature space infrastructure and resource exploitation underpin economic viability. 
The equator is the optimal location, and space debris removal is a critical safety priority. Space elevators can drastically cut orbital access costs 
and boost space industrial development. Their realization relies on material breakthroughs, phased verification and global cooperation, with projection 
for implementation in the late 21st century.

%地月运输
In the literature\upcite{ishimatsu2016generalized}, a generalized multi-commodity network flow (GMCNF) model has been proposed to optimize logistics within the Earth–Moon–Mars system.
 It incorporates multi-commodity flows like propellant, crew, and equipment, with in-situ resource utilization (ISRU) such as lunar water production2. Leveraging lunar resources reduces 
 total mass launched from Earth (TLMLEO) when combined with LOX/LH₂ propulsion and aerocapture2. Sensitivity analysis shows lunar ISRU becomes economically viable above 1.8 kg/year/kg productivity, 
 supporting sustainable cislunar supply chains2. This approach provides a foundational framework for efficient interplanetary transport networks2.

%火箭污染
A literatrue\upcite{ross2014radiative}shows:Rocket exhaust produces positive radiative forcing (RF), primarily due to black carbon (BC) and alumina particles.
Carbon dioxide (CO₂) emissions from rockets contribute negligibly to RF compared to particulate emissions.
Current global RF from rocket launches is estimated at 16 ± 8 mW m⁻², with BC accounting for 70 percents of this value.



\subsection{Our work}%-----------------------------------------------------------
We do such things ...
这部分直接上图

\begin{enumerate}[\bfseries 1.]
	\item We do ...
	\item We do ...
	\item We do ...
\end{enumerate}
%===========================第二部分==模型准备==========================================================
\section{Preparation of the Models}
\subsection{Assumptions and Explanations}
%为了简化问题,我们做出了以下假设,其中每一条都有对应的合理解释
To simplify the problem, we made the following assumptions, each of which has a corresponding reasonable explanation.
\begin{itemize}
	\item \textit{\textbf{Assumption 1:}}物资分级连续性:分为关键物资(优先SE运输)和普通物资,均视为准连续流体。
	\\$\hookrightarrow$ \textit{\textbf{Explanation:}}工程中物资存在优先级,SE高可靠性适配关键物资,符合调度逻辑。

	\item \textit{\textbf{Assumption 2:}}$\alpha(t)$ 为分段常数函数:将总工期划分为50个时间区间(每区间1-2年),每个区间 $\alpha_i$ 为常数。
	\\$\hookrightarrow$ \textit{\textbf{Explanation:}}平衡计算精度与效率,适配MCM竞赛求解需求,易实现且结果直观。

	\item \textit{\textbf{Assumption 3:}}水资源消耗:人均日用水量吨,循环效率 $\gamma=$。
	\\$\hookrightarrow$ \textit{\textbf{Explanation:}}参考国际空间站用水标准,循环效率符合2050年技术水平预测。

	\item \textit{\textbf{Assumption 4:}}太空电梯清洁能源使用率为,会随着技术进步不断升高直至100\%。
	\\$\hookrightarrow$ \textit{\textbf{Explanation:}}于绿色科学原理与题目场景合理设定

	\item \textit{\textbf{Assumption 5:}}非完美状态参数:TR发射失败概率\%,会随着技术的进步不断降低至阈值。
	\\$\hookrightarrow$ \textit{\textbf{Explanation:}}基于航天工程历史数据与题目场景合理设定。

	\item \textit{\textbf{Assumption 6:}}成本参数:SE固定成本亿美元,单位可变成本美元/吨;TR单次发射成本亿美元。
	\\$\hookrightarrow$ \textit{\textbf{Explanation:}}参考Falcon Heavy当前成本与未来技术迭代趋势,量级合理。

	\item \textit{\textbf{Assumption 7:}}顶端火箭发射失败时成本不计。
	\\$\hookrightarrow$ \textit{\textbf{Explanation:}}由于顶端火箭载荷量未知,成本波动计入原成本内。

	\item \textit{\textbf{Assumption 8:}}原有的火箭发射中心启用时成本为零。
	\\$\hookrightarrow$ \textit{\textbf{Explanation:}}现有发射场启用时不应有成本。
\end{itemize}
%这里只列出了主要的假设,其他假设会在专门的小节中单独讨论
Additional assumptions are made to simplify analysis for individual sections. These assumptions will be discussed at the appropriate locations.

\newpage
\subsection{Notations}%-----------------------------------------------------------------------------------
% 三线表(可以直接在excel里编辑好然后用excel2latex插件插入)

Table \ref{tb:notation} lists some important mathematical notations used in this paper.
\begin{table}[htbp]%----------------------------------------------
	\begin{center}
		\caption{Notations used in this paper}
		\begin{tabular}{cl}
			\toprule[1.5pt]
			\multicolumn{1}{m{4cm}}{\centering \textbf{Symbol}}
			                      & \multicolumn{1}{m{10cm}}{\textbf{ Description} }                       \\
			\midrule
			$\alpha_i$            & 第i个时间区间的SE运输比例( $\alpha(t)$ 分段常数)                      \\
			$q_i$                 & 第i个时间区间的平均运力                                                      \\
			$M_{tot}$             & 建设物资总量                                                           \\
			$\kappa_{SE,total}$   & SE系统总年运力                                                         \\
			$m_{load}$            & TR单次payload基准值                                                    \\
			$f_{TR}(j)$           & 第j个TR发射场年最大发射频次                                            \\
			$C_{fix}^{SE}$        & SE系统总固定成本                                                       \\
			$c_{var}^{SE}$        & SE单位可变成本+顶端火箭单位质量发射成本                                \\
			$c_{launch}$          & TR单次发射成本                                                         \\
			$P_{pop}$             & 月球殖民地人口                                                         \\
			$w_{per}$             & 人均日用水量                                                           \\
			$\gamma$              & 水资源循环效率                                                         \\
			$\eta_{clean}$        & 清洁能源使用率                                                         \\
			$Q_{SE}(i)$           & 前i个区间SE累积运输量                                                  \\
			$Q_{TR}(i)$           & 前i个区间TR累积运输量                                                  \\
			$D(i)$                & 第i个区间SE累积损坏因子                                                \\
			$\kappa_{SE}'(i)$     & 第i个区间SE有效运力(非完美状态)                                     \\
			$\kappa_{TR}'(i)$     & 第i个区间TR有效运力(非完美状态)                                     \\
			$f_{cost}$            & 全生命周期总成本                                                       \\
			$f_{time}$            & 总工期                                                                 \\
			$f_{env}$             & 综合环境影响指数                                                       \\
			\bottomrule[1.5pt]
		\end{tabular}\label{tb:notation}
		\begin{tablenotes}
			\footnotesize
			\item[*] *Some variables are not listed here and will be discussed in detail in each section. %此处加入注释*信息
		\end{tablenotes}
	\end{center}
\end{table}
\vspace{-1cm}%在\end{table}下加一行\vspace{-1cm} 其中-1的作用是缩短与下方文字距离的 切记!必须是负数



%数据处理------------------------------------------------------------------------


\subsection{Data}
\subsubsection{Data Collection}
%下面列出了我们收集数据的来源网站
Websites, where we collect data, are listed in Table \ref{tb:data}.

\begin{table}[htbp]%----------------------------------------------
	\begin{center}
		\caption{Data Sources used in this paper}
		\begin{tabular}{c c}
			\toprule[1.5pt]
			\multicolumn{1}{m{5cm}}{\centering \textbf{Database Names}}
			               & \multicolumn{1}{m{10cm}}{\centering \textbf{Database Websites}}   \\
			\midrule
			Semantic Scholar & \href{https://www.semanticscholar.org}{https://www.semanticscholar.org} \\
			NASA & \href{https://www.nasa.gov}{https://www.nasa.gov} \\
			USGS & \href{https://www.usgs.gov}{https://www.usgs.gov} \\
			Nature & \href{https://www.nature.com}{https://www.nature.com} \\
			UNFAO & \href{https://www.fao.org}{https://www.fao.org} \\
			MSCI & \href{https://www.msci-institute.com}{https://www.msci-institute.com} \\
			NREL & \href{https://docs.nrel.gov}{https://www.nrel.gov} \\
			\bottomrule[1.5pt]
		\end{tabular}\label{tb:data}
	\end{center}
\end{table}
\vspace{-1cm}%在\end{table}下加一行\vspace{-1cm} 其中-1的作用是缩短与下方文字距离的 切记!必须是负数
%\subsubsection{Data Processing}
%=================================第三部分====================================================================
\section{Model 1: 静态基准模型}
\subsection{模型概述}

静态基准模型在完美工作状态下分析三种运输场景,为比较不同物资运输方案的可行性、经济性和工期提供基础。

\subsection[场景一:仅SE模式]{场景一:仅SE模式}

\subsubsection{完美状态运力}

太空电梯系统的年运输能力为:
\begin{equation}
	\kappa_{SE,total}=179,000 \text{ 吨/年}
\end{equation}

\subsubsection{项目工期}

物资全部运输所需的总工期为:
\begin{equation}
	T_{SE}=\frac{M_{tot}}{\kappa_{SE,total}}=\frac{1 \times 10^8}{179000} \approx 558.7 \text{ 年}
\end{equation}

\subsubsection{总成本}

全生命周期成本包括固定基础设施成本和可变运营成本:
\begin{equation}
	C_{SE}=C_{fix}^{SE}+M_{tot} \cdot c_{var}^{SE}
\end{equation}

其中$C_{fix}^{SE}$为太空电梯系统的总固定成本,$c_{var}^{SE}$为单位可变成本(美元/吨)。

\subsection[场景二:仅TR模式]{场景二:仅TR模式}

\subsubsection{完美状态运力}

启用全球10个发射场时,传统火箭系统的年运输能力为:
\begin{equation}
	\kappa_{TR}=10 \cdot f_{TR}(j) \cdot m_{load}=10 \times 20 \times 125=25,000 \text{ 吨/年}
\end{equation}

其中$f_{TR}(j)$为单个发射场年最大发射频次(20次/年),$m_{load}$为单次发射的基准载荷(150吨)。

\subsubsection{项目工期}

物资全部运输所需的总工期为:
\begin{equation}
	T_{TR}=\frac{M_{tot}}{\kappa_{TR}}=\frac{1 \times 10^8}{25000}=4000 \text{ 年}
\end{equation}

\subsubsection{总成本}

成本由所需发射次数决定:
\begin{equation}
	C_{TR}=\frac{M_{tot}}{m_{load}} \cdot c_{launch}
\end{equation}

其中$c_{launch}$为单次发射成本(美元/次)。

\subsection[场景三:混合静态模式]{场景三:混合静态模式}

混合策略结合太空电梯和传统火箭两种运输方式,以优化项目工期和成本。

\subsubsection{项目工期}

混合运输的工期由两种运输方式中耗时较长的决定:
\begin{equation}
	T_C(\alpha)=\max\left( \frac{\alpha M_{tot}}{\kappa_{SE,total}}, \frac{(1-\alpha) M_{tot}}{\kappa_{TR}} \right)
\end{equation}

%3.5
\subsubsection{最优分配比例}

从工期角度考虑,最优分配比例$\alpha^*$满足两种运输方式的完成时间相等:
\begin{equation}
	\frac{\alpha^* M_{tot}}{\kappa_{SE,total}}=\frac{(1-\alpha^*) M_{tot}}{\kappa_{TR}}
\end{equation}

解得最优分配比例:
\begin{equation}
	\alpha^*=\frac{\kappa_{SE,total}}{\kappa_{SE,total}+\kappa_{TR}}=\frac{179000}{179000+25000} \approx 0.417
\end{equation}

\subsubsection{总成本}

混合策略的全生命周期成本为:
\begin{equation}
	C_C=\frac{(1-\alpha) M_{tot}}{m_{load}} \cdot c_{launch}+C_{fix}^{SE}+\alpha M_{tot} \cdot c_{var}^{SE}
\end{equation}

该成本包括太空电梯固定基础设施成本(一次性)、太空电梯运营可变成本和传统火箭发射成本。

\subsection{三种场景的比较分析}

静态基准模型提供以下重要洞察:

\begin{itemize}
	\item \textbf{仅SE模式}:需约558.7年完成物资运输,一旦基础设施建成后,运营成本最低。
	
	\item \textbf{仅TR模式}:需约4000年完成物资运输,在项目工期和成本约束下经济上不可行。
	
	\item \textbf{混合最优模式}($\alpha^* \approx 0.417$):采用最优分配比例,工期约为$\max\left(\frac{0.417 \times 10^8}{179000}, \frac{0.583 \times 10^8}{25000}\right) \approx 232$年,在基础设施固定成本和按次发射成本间实现平衡。
	
	\item \textbf{决策选择}:三种方案的优劣取决于对工期最小化或成本最小化的优先级权重。
\end{itemize}

\begin{figure}[htbp]
	\centering
	\begin{subfigure}[b]{1\textwidth} %设置缩放比例,这里的.5代表缩放为原来的50%
		\includegraphics[width=\textwidth]{img/Figure_35.png}
		\caption*{}
		\label{}
	\end{subfigure}
	\caption{35}\label{subfigure}
\end{figure}
\newpage
%===========================================第四部分====================================================================
\section[Model 2: Dynamic Extended Model]{Model 2: 动态扩展模型(时间自适应策略)}

\subsection[Dynamic Adjustment Logic]{动态调整逻辑}

由于工期极长,随着材料科学、能源系统和自动化维护技术的提升,若仍以静态的$\alpha$案进行规划,将无法反映运输方案在不同阶段的经济性演化,导致早期或后期资源配置严重失衡。随着时间的推移和技术的发展,太空电梯系统的单位可变成本$c_{var}^{SE}$逐渐下降。采用如下理想化模型表示这一过程:

\begin{equation}
	c_{var}^{SE}=0.8c_0+0.2c_0\exp(-0.0139i)
\end{equation}

其中$c_0$为初始成本,25个时间区间后成本降至原值的0.9倍。

%4.2

\subsection{动态目标函数}

\subsubsection{目标1:总项目工期}

总工期是使累积运输量达到要求的最小时间区间数:

\begin{equation}
	T= \min \left\{ k \cdot \Delta t \mid \sum_{i=1}^k q_i \geq M_{tot} \right\}
\end{equation}

其中累积运输量递推关系为:
\begin{align}
	Q_{SE}(i) &= Q_{SE}(i-1)+\alpha_i \cdot q(i) \cdot \Delta t \\
	Q_{TR}(i) &= Q_{TR}(i-1)+(1-\alpha_i) \cdot q(i) \cdot \Delta t
\end{align}

时间区间长度$\Delta t = 10$年,可根据需要调整。

\subsubsection{目标2:全生命周期成本(分段求和)}

全生命周期总成本为:

\begin{equation}
	f_{cost} = \mathbb{I}\left( \sum_{i=1}^{k} \alpha_i > 0 \right) \cdot C_{fix}^{SE} + \sum_{i=1}^{k} \left[ q_{SE}(i) \cdot c_{var}^{SE}(i) + q_{TR}(i) \cdot c_{TR}(i) \right]\cdot \Delta t
\end{equation}

其中:
\begin{equation}
	c_{TR}(i)=\frac{c_{launch}}{m_{load}}
\end{equation}

指示函数$\mathbb{I}$确保太空电梯固定成本仅在使用太空电梯时计入一次。

\subsection{动态约束条件}

优化问题受以下约束条件限制:

\begin{equation}
	\begin{cases}
		0 \leq \alpha_i \leq 1, \quad \forall i=1,2,\ldots,50 \\[6pt]
		\alpha_i \cdot q(i) \leq \kappa_{SE}'(i), \quad \forall i \\[6pt]
		(1-\alpha_i) \cdot q(i) \leq \kappa_{TR}'(i), \quad \forall i \\[6pt]
		Q_{SE}(50) + Q_{TR}(50) \geq M_{tot}
	\end{cases}
\end{equation}

约束条件的含义分别为:

\begin{itemize}
	\item 分配比例$\alpha_i$必须在有效范围内
	
	\item 每个时间区间太空电梯的运输量不能超过其有效运力$\kappa_{SE}'(i)$
	
	\item 每个时间区间传统火箭系统的运输量不能超过其有效运力$\kappa_{TR}'(i)$
	
	\item 第50个区间结束时,累积运输总量必须达到所需的全部物资
\end{itemize}

\begin{figure}[htbp]
	\centering
	\begin{subfigure}[b]{.8\textwidth} %设置缩放比例,这里的.5代表缩放为原来的50%
		\includegraphics[width=\textwidth]{img/Figure_41&42.png}
		\caption*{}
		\label{}
	\end{subfigure}
	\caption{43}\label{subfigure}
\end{figure}

%4.4================================动态模型特点====================================================================
\subsection{模型特点}

动态$\alpha(t)$扩展模型相比静态基准模型的主要优势为:

\begin{enumerate}
	\item \textbf{时间自适应}:根据技术发展动态调整运输方式比例,充分利用成本下降机遇
	
	\item \textbf{成本优化}:通过实时比较两种运输方式的单位成本,在成本均衡区间内进行精细优化
	
	\item \textbf{分段规划}:将总工期划分为50个区间,每个区间独立决策,便于实施和调整
	
	\item \textbf{多目标平衡}:同时考虑工期、成本和环境影响,寻求综合最优方案
\end{enumerate}

\begin{figure}[htbp]
	\centering
	\begin{subfigure}[b]{1\textwidth} %设置缩放比例,这里的.5代表缩放为原来的50%
		\includegraphics[width=\textwidth]{img/Figure_44.png}
		\caption*{}
		\label{}
	\end{subfigure}
	\caption{44}\label{subfigure}
\end{figure}

% 长表格示例,更多用法请参考 longtable 宏包文档
% 以下环境及对应参数可实现表格内的自动换行与表格的自动断页
% 您也可以选择自行载入 tabularx 宏包,并通过 X 参数指定对应列自动换行

%\section{Model 3: 随机扰动的修正模型}

\subsection{随机扰动下的模型修正}

\subsubsection{火箭发射失败的修正}

\paragraph{失败概率与有效载荷}

传统火箭系统存在发射失败的风险。引入发射失败概率$P_{\mathrm{fail}}$,有效载荷的期望值为:

\begin{equation}
	E[m] = m_{\mathrm{load}} \cdot (1 - P_{\mathrm{fail}})
\end{equation}

\paragraph{发射成本调整}

考虑失败风险导致的成本增加:

\begin{equation}
	c'_{\mathrm{launch}} = c_{\mathrm{launch}} + P_{\mathrm{fail}} \cdot (c_{\mathrm{launch}} + c_{\mathrm{cargo\_loss}})
\end{equation}

其中$c_{\mathrm{cargo\_loss}}$为货物损失成本。
$c_{cargo\_loss}$ 无法查询到准确信息,通过搜集资料触发失败后一般是提供一次\textbf{Reflight(再飞)且客户无需再支付该次发射服务费}(相当于“用服务替代现金赔付”,覆盖的是发射服务成本),也就是可以认为上面公式可以写成:

\begin{equation}
	c'_{\mathrm{launch}} = c_{\mathrm{launch}} + P_{\mathrm{fail}} \cdot c_{\mathrm{launch}} = c_{\mathrm{launch}} (1 + 2 P_{\mathrm{fail}})
\end{equation}

\paragraph{修正运力}

火箭系统的实际运力上限应为:
\begin{equation}
	\kappa'_{TR}(i) = \kappa_{TR} \cdot (1 - P_{\mathrm{fail}}(i))
\end{equation}

其中 $P_{\mathrm{fail}}(i)$ 为经15个时间间隔(即150年)从 1\% 下降至 0.1\% 的学习曲线,呈负指数形状。
\begin{equation}
	P_{\mathrm{fail}}(i) = 0.01 \cdot \exp(-\frac{\ln(10)}{15} \cdot i)
\end{equation}


\paragraph{修正成本公式}

成本公式中,TR 的单位成本 $c_{TR}(i)$ 应改为:
\begin{equation}
	c_{TR}(i) = \frac{c'_{\mathrm{launch}}}{m_{load}} = \frac{c_{\mathrm{launch}} (1 + 2 P_{\mathrm{fail}}(i))}{m_{load}}
\end{equation}

这样将发射失败的风险成本纳入总成本计算。

\subsubsection{系绳摇晃的科里奥利力与风切变扰动}

\paragraph{瞬时可用度函数}

太空电梯系统的可用性受摆动幅度和风速的影响。定义瞬时可用度函数为:

\begin{equation}
	\eta_{SE}(t) = 
	\begin{cases}
		1, & \text{if } \theta(t) < \theta_{\mathrm{crit}} \text{ and } W_{\mathrm{wind}}(t) < v_{\mathrm{safe}} \\
		0, & \text{otherwise}
	\end{cases}
\end{equation}

其中系绳横向受风截,通过物理低阶近似,我们不妨将偏转角拟合为
\[
\theta(t) = k \cdot v_{wind}(t)^2
\]

通过查找资料,常规地面系绳可拟合出 $k$ 约为 0.019,来自 \href{https://www3.eng.cam.ac.uk/~hemh1/SPICE/papers/Coulombe-Pontbriand_2005_MastersThesis.pdf}{Simulation and Experimental Validation of Tethered Aerostat Model},但是显然不符合当前系绳材质与高空环境,当前为强约束/高张力/高抗风设计,通过查找资料,可以确定 $\theta_{\mathrm{crit}}$ 为 $0.1\ \mathrm{rad}$ ($5.73^\circ$), $v_{\mathrm{safe}}$ 确定为 $15\ \mathrm{m/s}$,进而反解出当前 $k$ 为 $4.44\times10^{-4}$,即当 $v_{wind}(t) = v_{\mathrm{safe}}$ 时,$\theta(t) \approx \theta_{\mathrm{crit}}$,设计点是一致的。可以合并为单一阈值影响,故下文仅讨论风速的影响。

\begin{equation}
	\eta_{SE}(t)=
	\begin{cases}
	1, & \text{if }  v_{\mathrm{wind}}(t) < v_{\mathrm{safe}} \\
	0, & \text{otherwise}
	\end{cases}
\end{equation}

其中风场满足参数为$k=3$,$\lambda=8$的韦布尔分布。其次同时还要建立年代系统受风影响的的状态转变,将风环境抽象为两类状态:

\begin{itemize}
	\item \textbf{Normal State (正常状态)}:可用度围绕一个低频背景缓慢波动;
	\item \textbf{Extreme State (极端状态)}:代表长期风暴年代/异常环流时期,可用度显著降低并持续若干区间。
\end{itemize}


其中极端期的开始阶段建模为泊松过程,持续过程建模为指数分布

先建立低频背景可用度年代可变率:
\begin{equation}
	\eta_\mathrm{base}(i)=\mu+A\sin\left(\frac{2\pi(i-1)}{P}\right)
\end{equation}

建立极端期示性变量$I_{ext}$表示是否处于极端状态
\begin{equation}
	\bar{\eta}_{SE}(i)=\mathrm{clip}\left(\eta_{\mathrm{base}}(i)-\Delta\cdot I_{\mathrm{ext}}(i),0,1\right)
\end{equation}

其中 clip 是为了保证 [0,1]合理范围。

\paragraph{顶端火箭发射失败概率}

由于太空电梯顶端仍需火箭发射将货物送入轨道,存在失败概率$P_{\mathrm{fail}}'$。

\paragraph{修正后的年有效运力}
综合考虑摆动限制、维护停机和火箭发射失败,太空电梯的修正后年有效运力为:

\begin{equation}
	\kappa_{SE}^{\prime}=\kappa_{SE}\cdot\bar{\eta}_{SE}\cdot(1-\beta_{\mathrm{maint}})\cdot(1-P_{\mathrm{fail}})
\end{equation}

其中:
\begin{itemize}
	\item $\bar{\eta}_{SE}$为太空电梯系统的平均可用度
	\item $\beta_{\mathrm{maint}}$为维护停机系数,在本工作中维护停机系数简化为$\beta_{maint}=0$
	\item $P_{\mathrm{fail}}$为顶端火箭发射失败概率,取常数$P_{fail}=0.005$
\end{itemize}

\subsubsection{模型求解}
基于上述修正,动态扩展模型的实际运力和实际成本作出相应调整,并通过求解得到时间自适应的最优运输比例$\alpha_i$如下图所示:

\subsection{一年用水需求下的运输成本}

\subsubsection{月球殖民地年净用水需求}

月球殖民地一年的净用水需求(考虑水资源循环)为:

\begin{equation}
	M_{net}=M_{gross}\cdot(1-\gamma)=365P_{pop}\cdot w_{per}\cdot(1-\gamma)
\end{equation}

其中:
\begin{itemize}
	\item $M_{gross}$为年总用水量
	\item $\gamma=$为水资源循环效率
	\item $P_{pop}=10$万人为月球殖民地人口
	\item $w_{per}=$吨/(人·天)为人均日用水量
\end{itemize}

\subsubsection{额外运输成本}

在此视角下,不同策略的成本如下:
\begin{equation}
    C_{SEonly}=M_{net}c_{var}^{SE}=159.7 \text{ Billion USD}  
\end{equation}
\begin{equation} 
	C_{TRonly}=M_{net}c'_{launch}/m_{load}=417.2 \text{ Billion USD}  
\end{equation}
\begin{equation} 
	 C_{static}=\alpha\cdot M_{net}c_{var}^{SE}+(1-\alpha)M_{net}c'_{launch}/m_{load}= 224.1 \text{ Billion USD} 
\end{equation}
    
在工期尾部阶段,动态模型的$\alpha_i$取值基本为1,其成本
\begin{equation}
	C_{dynamic}=C_{SEonly}=159.7 \text{ Billion USD}
\end{equation}
    
\subsection{环境影响模型}
\subsubsection{环境影响系数的建立}
TR-only模型的主要环境污染在于火箭发射产生的二氧化碳排放 以及黑碳和氧化铝对臭氧层的破坏,这些排放量均与火箭发射次数成正比,\textbf{环境影响系数}为:

\[ E_{TR-only} = M_{tot} \cdot (\omega_{CO_{2}}\cdot \epsilon_{CO_{2}}+\omega_{BC}\cdot \epsilon_{BC}+\omega_{Al_2 O_3}\cdot \epsilon_{Al_2 O_3})/m_{load}\]

SE-only模型的主要环境污染在于运送货物时耗费的电力,这些电力可能来自于非清洁能源,在生产电力的过程中造成污染,\textbf{环境影响系数}为:

\[ E_{SE-only} = M_{tot} \cdot e_{per}\cdot (1-\eta_{clean})\cdot\epsilon_{coal}\]
其中$\eta_{clean}$为清洁能源的使用率,$\epsilon_{coal}$为燃煤发电时每千瓦时释放的CO$_2$。

静态混合模型的环境污染同时来自于两种污染,\textbf{环境影响系数}为:
\begin{equation}
\begin{split}
	E_{static} = &M_{tot}\cdot (1-\alpha) \cdot (\omega_{CO_{2}}\cdot \epsilon_{CO_{2}}+\omega_{BC}\cdot \epsilon_{BC}+\omega_{Al_2 O_3}\cdot \epsilon_{Al_2 O_3})/m_{load}\\
	&+{M_{tot}\cdot \alpha \cdot e_{per}\cdot (1-\eta_{clean})}\cdot\epsilon_{coal}
\end{split}
\end{equation}
而动态模型的\textbf{环境影响系数}类似于上面的式子:
\begin{equation}
\begin{split}
	E_{dynamic} = &\sum_{i=1}^{K} \left[ (1-\alpha_i) \cdot (\omega_{CO_{2}}\cdot \epsilon_{CO_{2}}+\omega_{BC}\cdot \epsilon_{BC}+\omega_{Al_2 O_3}\cdot \epsilon_{Al_2 O_3})/m_{load} \right.\\
	&\left. +{\alpha_i \cdot e_{per}\cdot (1-\eta_{clean})}\cdot\epsilon_{coal} \right] \cdot q(i) \cdot \Delta t
\end{split}
\end{equation}

其中$\eta_{clean}$考虑到未来清洁能源比例的提升,假设其随时间线性增长,从初始的0.3增长到1,历时30个时间间隔,也即300年后达到100\%清洁能源。

\subsubsection{环境影响系数对最优策略的影响}
TR-only和SE-only模型的策略不会发生变化,各自的环境影响系数为:
\[E_{TR-only}=54.407B\]
\[E_{SE-only}=-3B t + 1.05T\]

而静态混合模型和动态模型在考虑环境影响系数后,最优策略\textbf{发生变化},分别求解得:


%===========================================第五部分====================================================================
\newpage
\section{上述模型的求解}
为解决上述非线性、多目标、多阶段的优化问题,我们采用逐阶段扩展搜索算法(Stage-wise Extended Search Algorithm, SESA)。该算法结合动态规划和启发式搜索的优势,适用于复杂的决策问题。并采用贪心变动成本法(Greedy Marginal Cost Method, GMCM)对模型进行求解和优化。为求解多目标优化问题,我们采用帕累托前沿分析(Pareto Front Analysis)方法,识别在工期、成本和环境影响之间的最佳权衡方案。
\subsection{决策变量}

算法中的核心决策变量为各阶段、各技术模式下的产能分配量。具体而言:
\begin{itemize}
	\item \( x_{i}^{SE} \):第 \(i\) 阶段采用 SE 技术分配的产能(吨)。
	\item \( x_{i}^{TR} \):第 \(i\) 阶段采用 TR 技术分配的产能(吨)。
\end{itemize}

这些变量受各阶段技术对应的最大产能容量约束限制,即:

\begin{equation}
0 \leq x_{i}^{SE} \leq C_{i}^{SE}, \quad 0 \leq x_{i}^{TR} \leq C_{i}^{TR}
\end{equation}

满足总需求约束:
\begin{equation}
\sum_{i=1}^{K} (x_{i}^{SE} + x_{i}^{TR}) \geq M_{tot}
\end{equation}

其中,\(K\) 表示考虑的最大时间阶段数量,动态变化。
\subsection{逐阶段扩展搜索算法}

算法通过从早期阶段开始,逐步增加考虑的时间区间长度,实现对动态产能建设计划的搜索:

\begin{enumerate}
	\item \textbf{阶段容量累积验证}:在每一轮迭代中,计算截止当前阶段的总可用产能(SE和TR两种模式)之和,判断是否足以满足总需求,若不足则直接跳过,避免无效计算。
	\item \textbf{方案选项构造与排序}:对每个阶段的两种技术产能与对应单位变动成本进行整合,形成供分配的选项集合,按单位成本升序排列,确保优先利用成本较低的产能资源。
	\item \textbf{贪心填充分配}:按照排序后的选项,逐项分配产能以满足当前阶段的需求,优先消耗低成本产能,直至需求满足或资源耗尽。
	\item \textbf{混合技术约束处理}:在分配过程中,若需保证两种技术均有投入,先为每种技术预留最小需求量,之后对剩余需求执行贪心分配,确保方案符合混合使用的业务规则。
	\item \textbf{成本累计和记录}:计算当前阶段的总变,加上可能存在的固定成本,保存当前阶段对应的可行方案及成本信息,供后续筛选。
\end{enumerate}
这种逐阶段动态搜索方式兼顾了时间进度与资源约束,逐步逼近满足需求的最优方案空间。

\subsection{多目标前沿判定}

为辅助决策,算法在时间和成本两个目标维度上筛选关键方案,具体实现包括:
\begin{enumerate}
	\item \textbf{帕累托前沿筛选}:从所有可行方案中筛选不被其他方案在成本和时间上同时优越的非劣解,形成方案的 Pareto 前沿,保证所选方案在一定的时间范围内成本最优。
	\item \textbf{Knee 点识别}:对 Pareto 前沿方案进行归一化处理,将时间和成本标准化至相同尺度后,计算曲线与两端连接直线的垂直距离。最大距离点即为 Knee 点,通常代表成本效益显著改善的折中方案。
	\item \textbf{关键方案输出}:同时记录最短可行方案(Star 点),用于比较Knee点与Star点的差异,为决策者提供更全面的参考信息。
	\item \textbf{可视化展示}:将帕累托前沿以图形方式展示,直观反映不同方案在多目标空间中的分布情况。
\end{enumerate}





\newpage

Respected Lunar Base Management (LBM) Agency:
Greetings! Regarding your agency's planned construction and operation task of a 100,000-person lunar base to be launched in 2050, based on multi-dimensional modeling analysis and feasibility demonstration, we aim to recommend a comprehensive action plan that balances efficiency, economy, and robustness to contribute to the success of the task.
Firstly, through modeling analysis, a single-mode transportation scheme would result in a relatively long construction period. The optimal transportation strategy is a hybrid dynamic transportation mode, wherein 41.7% of construction materials are transported via SE and 58.3% via TR. This approach can shorten the total construction period to approximately 232 years while achieving a significant cost reduction compared to the sole use of TR. Furthermore, a dynamic adjustment mechanism has been established, incorporating a time-adaptive dynamic expansion model. The total construction period is divided into multiple time intervals, and the α(i) allocation ratio is real-time optimized to maximize the cost advantages brought by technological iteration.
Secondly, considering the potential failure risks of the transportation system, the proposed plan includes correction strategies. On one hand, accounting for the launch failure probability of TR, the model reliability is ensured through effective payload expectation calculation, risk cost superposition, and transportation capacity reduction. On the other hand, addressing the tether sway control of SE, a Weibull distribution wind field model is adopted to set critical swing angles and safe wind speed thresholds. Combined with the maintenance downtime coefficient and the launch success rate of the top rocket, the annual transportation capacity of SE remains stable after correction.
Thirdly, in accordance with the water use standards of the International Space Station and 2050 technological forecasts, the annual net water demand of the 100,000-person base is [to be supplemented]. We recommend adopting a "transportation + recycling" mode, integrating the net water demand into the hybrid transportation plan, and calculating the additional transportation costs to achieve cost reduction and efficiency improvement.
Finally, considering the potential impacts of the overall plan on the Earth's ecological environment, the optimal comprehensive objective function is achieved through Pareto frontier analysis and Knee point identification.
The proposed plan boasts the following core advantages: Firstly, strong feasibility— the hybrid dynamic mode breaks through the efficiency bottleneck of a single transportation method; Secondly, high robustness— the correction model incorporating risk factors ensures uninterrupted transportation tasks under extreme scenarios; Thirdly, excellent sustainability— ecological impacts are actively considered, aligning with the concept of low-carbon space development.
This plan is derived from mathematical modeling and validated through multiple rounds of sensitivity and robustness tests to ensure scientific reliability. If your agency requires further refinement of the model, please feel free to contact us. We will make every effort to contribute to the cause of human space exploration.
%===============================================第六部分=============================================
\section{Test the Model}
\subsection{Sensitivity  Analysis}
敏感性分析目标:  
\begin{itemize}
	\item 量化参数对目标函数的影响程度;
	\item 识别关键参数,指导决策和优化;
\end{itemize}
\subsubsection{选取参数}
根据模型,选取以下参数进行敏感性分析:
\begin{itemize}
	\item SE系统年运力 $\kappa_{SE,total}$
	\item TR单次有效载荷 $m_{load}$
	\item SE缆绳摆动导致的可用度 $\eta_{SE}$ 
	\item 成本参数 $c_{var}^{SE}$、$c'_{launch}$
	\item 清洁能源使用率 $\eta_{clean}$
\end{itemize}
\subsubsection{分析方法}
采用\textbf{单因素敏感性分析} + \textbf{多因素交互分析}:
\begin{itemize}
	\item \textbf{单因素}:每次只改变一个参数的值(±5\%、±10\%、±20\%),保持其他参数不变,计算目标函数的变化率:
\[
S_x = \frac{\Delta f}{f} \cdot \frac{x}{\Delta x}
\]
其中 $f$ 是目标函数值(工期、成本),$x$ 是参数值。
	\item \textbf{多因素}:用拉丁超立方抽样(LHS)方法,在参数的可能范围内随机采样,运行模型,计算 Sobol 灵敏度指数,分析交互效应。
\end{itemize}

\subsubsection{结果展示}
\subsection{Robustness Analysis}

鲁棒性分析目标:
\begin{itemize}
	\item 判断在输入参数有不确定性波动时,原方案目标值产生的误差大小;
	\item 确定哪种运输策略(SE-only、TR-only、混合模式、动态模式)对扰动最不敏感。
\end{itemize}

\subsubsection{不确定性建模}
根据原有参数,可以有以下不确定性:
\begin{itemize}
	\item \textbf{火箭失败概率波动}:设为 Beta 分布(偏向平均值但存在极端值处的风险)。
	\item \textbf{成本波动}:假设未来价格受通胀和技术进步影响,成本服从正态分布。
	\item \textbf{技术改进速度波动}:$c_{var}^{SE}$公式中的参数服从正态分布,影响 $c_{var}^{SE}$ 的下降速率。
\end{itemize}
\subsubsection{鲁棒性评估指标}
\begin{itemize}
	\item 目标函数偏离度:衡量扰动后目标函数值相对于基准值的变化程度:\[
	R = \frac{|f_{\mathrm{perturbed}} - f_{\mathrm{baseline}}|}{f_{\mathrm{baseline}}}
	\]
	\item 最坏情况性能下降幅度:取样本中最差5\%的结果,用于风险评估。
\end{itemize}

\subsubsection{分析方法}
先用原始参数求出基准工期$T_0$,基准成本$f_0$,基准环境影响$E_0$。
然后进行扰动后仿真:
\begin{itemize}
	\item 随机抽样不同扰动组合:
	\item 采用原始参数求出的方案对每个组合运行模型,记录各指标。
	\item 计算偏离度和最坏情况性能下降幅度:
\end{itemize}

\subsubsection{结果展示}
\begin{itemize}
	\item 各模型的目标偏离度排名
	\item 各模型的最坏情况性能下降幅度排名(其实基本就是鲁棒性排名)
\end{itemize}
\texttt{这部分很重要,不能缺!}








%==============================================第七部分================================================
\section{Conclusion}
\subsection{Summary of Results}
\begin{table}[htbp]
  \centering
  \caption{results}
  \label{tab:results}
  \begin{tabular}{lcccc}
    \toprule
    参数 & TR-Only & SE-Only & Static Mixing & Dynamic Mixing \\
    \midrule
    $\alpha \in (0,1)$ & 0 & 1 & Constant & Constant on 50 Intervals \\
    Durations(year) & & & & \\
    Cost(BD) & & & & \\
    Cost of maintaining water supply(BD) & & & & \\
    Environmental Impact & 4 & 3 & 2 & 1 \\
    Robustness(rank) & 4 & 3 & 2 & 1 \\
    \bottomrule
  \end{tabular}
\end{table}
\begin{table}[htbp]
  \centering
  \caption{sensitivity} % 可自定义表格标题
  \label{tab:sensitivity} % 可自定义引用标签,用于\ref{}引用
  \begin{tabular}{lcccc} % 列格式:l=左对齐(参数列),cccc=居中对齐(4个对比列)
    \toprule % 顶部粗线
    参数 & TR-Only & SE-Only & Static Mixing & Dynamic Mixing \\
    \midrule % 中间细线
    $c_{launch}$ & +++ & $\pm$ & ++ & + \\
    $\kappa_{se\_total}$ & - - & ++ & - & - - \\
    $\eta_{se}$ & - & ++ & - & - - \\
    $m_{load}$ & $\pm$ & $\pm$ & - - - & - - \\
    $\eta_{clean}$ & $\pm$ & ++ & - & - - \\
    $\text{c}_{var}^{SE}$ & $\pm$ & $\pm$ & - & + \\
    \bottomrule % 底部粗线
  \end{tabular}
\end{table}
\subsection{Strengths}%------------------优点----------------
\begin{itemize}
	%1. 有灵敏度分析与稳健性分析
	\item The sensitivity analysis of the model demonstrates the effectiveness of the model under different parameter combinations and prove the robustness of the mod
	\item Second one ...
\end{itemize}

\subsection{Weaknesses and Improvements}%---------------缺点与改进---------------------
\begin{itemize}
	\item The analysis of fish migration can be more accurate if we have more complete data;
	\item Some approximate analysis methods are applied to model the management of fishing
	      companies, which may lead to a situation contrary to the actual one  in extreme cases.
	\item 此处引用了LLM.\upcite{openai2024chatgpt}
\end{itemize}




% 以下为信件/备忘录部分,不需要可自行去掉========================================================================
% 如有需要可将整个 letter 环境移动到文章开头或中间
% 请在第二个花括号内填写标题,如「信件」(Letter)或「备忘录」(Memorandum)
\begin{letter}{Memorandum}
	\begin{flushleft}  % 左对齐环境,无首行缩进
		\textbf{To:} Heishan Yan\\
		\textbf{From:} Team 1234567\\
		\textbf{Date:} October 1st, 2019\\
		\textbf{Subject:} A better choice than MS Word: \LaTeX
	\end{flushleft}

	In the memo, we want to introduce you to an alternate typesetting program to the prevailing MS Word: \textbf{\LaTeX}. In fact, the history of \LaTeX\ is even longer than that of MS Word. In the 1970s, the famous computer scientist Donald Knuth first came out with a typesetting program,  named \TeX\ \ldots

	Firstly, \ldots

	Secondly, \ldots

	Lastly, \ldots

	According to all those mentioned above, it is really worth to have a try on \LaTeX!
\end{letter}

%=================================================================================================================




% 参考文献,直接把bib格式粘贴到References.bib里面,此处无需改动!!!!!!!!!!!!!!!!!!
\bibliographystyle{unsrt} %规定了参考文献的格式
\begin{center}
	\bibliography{references.bib} %调出LaTeX生成参考文献列表
\end{center}
%=====================================================================================



% 以下为附录内容
% 如您的论文中不需要附录,请自行删除
\begin{subappendices}  % 附录环境开始
	\section{Appendix C: Report on Use of AI}
	% 人工智能的报告
	1. OpenAI ChatGPT (Nov 5, 2023 version, ChatGPT-4)

	Query1: <insert the exact wording you input into the AI tool>
	
	Output: <insert the complete output from the AI tool>

	2. OpenAI Ernie (Nov 5, 2023 version, Ernie 4.0)
	
	Query1: <insert the exact wording of any subsequent input into the AI tool>
	
	Output: <insert the complete output from the second query>
	
	3. Github CoPilot (Feb 3, 2024 version)
	
	Query1: <insert the exact wording you input into the AI tool>
	
	Output: <insert the complete output from the AI tool>
	
	4. Google Bard (Feb 2, 2024 version)
	
	Query: <insert the exact wording of your query>
	
	Output: <insert the complete output from the AI tool>
	

\end{subappendices}  % 附录内容结束

\end{document}  % 结束
